\documentclass{article}

\usepackage{amsmath}
\usepackage{graphicx}

\title{Cahier de laboratoire}
\date{Le 22 Février 2024}
\author{Cătălin Bozan, Liviu Arsenescu}
\renewcommand{\contentsname{Contenu}}

\begin{document}
    \pagenumbering{gobble}
    \maketitle
    \newpage

    \section*{Résumé}
    Section for Abstract
    \newpage

    \tableofcontents
    \newpage
    \pagenumbering{arabic}

    \section{Rendu 1 (Laboratoire 1 : 22.02.2024)}
    \subsection{Tâches à effectuer}
    Analyser le projet Vivado fourni et créer les documents suivants:
    \begin{enumerate}
        \item Schéma bloc du processeur (nanoProcesseur)
        \item Schéma bloc du contrôleure (nanoControleur)
        \item Graph des états du séquenceur
        \item Table de vérité des sorties du séquenceur
        \item Plan mémoire
    \end{enumerate}
    \subsection{Parcours}
    Voici notre parcours, pour résoudre les tâches:
    \begin{itemize}
        \item \textbf{22.02.2023:}
        \begin{enumerate}
            \item Nous avons configuré un repository GitHub partagé, pour gérer facilement le projet, et avoir une gestion des versions si quelque chone ne va pas.
            \item Nous avons regardé le projet initial, analysé et discuté chaque composant du processeur.
            \item Nous avons choisi les outils que nous utiliserons pour le projet (LaTeX pour la documentation, draw.io pour dessiner les schémas).
            \item Nous avons réparti les tâches pour le prochain laboratoire.
        \end{enumerate}
        Ces tâches ont duré environ 2 heures et ont été réalisées en équipe.
        \item \textbf{29.02.2023:}
        \begin{enumerate}
            \item \textit{\underline{Ensemble:}} Nous avons discuté en détail les fichiers .vhd, pour pouvoir faire les schémas.
            \item \textit{\underline{Liviu:}} J'ai commencé à faire le schéma du nanoProcesseur.
            \item \textit{\underline{Cătălin:}} J'ai commencé à faire le schéma pour le nanoControleur.
            \item \textit{\underline{Ensemble:}} Nous avons écrit la documentation.
        \end{enumerate}
        Nous avons travaillé pendant environ 2 heures et demie, et les tâches ont été réparties comme suit.
        \item \textbf{07.03.2024:}
        \begin{enumerate}
            \item \underline{Liviu:} J'ai encore travaillé sur la chéma du nanoProcesseur.
            \item \underline{Cătălin:} J'ai encore travaillé sur la chéma du nanoControleur.
        \end{enumerate}
        Nous avons travaillé pendant environ 2 heures et demie, et les tâches ont été réparties comme suit.
        \item \textbf{14.03.2024:}
        \begin{enumerate}
            \item \underline{Liviu:} J'ai fini le schéma du nanoProcesseur
            \item \underline{Cătălin:} J'ai fini le schéma du nanoControleur
            \item \underline{Cătălin:} J'ai fini le schéma d'états pour le sequenceur
        \end{enumerate}
        Nous avons travaillé pendant environ 2 heures et demie, et les tâches ont été réparties comme suit.
        \item \textbf{15.03.2024:} \\
        Après avoir réalisé le schéma, on obtient le tableau suivant pour la machine à états:
        \begin{center}
            \begin{tabular}{|c|c|c|c|c|}
                \hline
                État & PC\_load\_i & IR\_load\_i & data\_wr\_o & reset\_i \\
                \hline
                \hline sRESET & '-' & '-' & '-' & '0' \\
                \hline sIR\_LOAD a & '0' & '0' & '1' & '1' \\
                \hline sIR\_DECODE & '0' & '1' & '0' & '1' \\
                \hline sOPCODE\_DECDE & '1' & '0' & '0' & '1' \\
                \hline
            \end{tabular}
        \end{center}
        Pour la memory map, on a analysé Address\_Decode.vhd et RAM.vhd, et on obtient le tableau suivant:
        \begin{center}
            \begin{tabular}{|c|c|c|}
                \hline
                Application & première adresse & dernière adresse \\
                \hline
                \hline RAM adresses & 0xe0 & 0xff \\
                \hline Port a & 0x10 & - \\
                \hline Port b & 0x11 & - \\
                \hline
            \end{tabular}
        \end{center}
        Aujourd'hui, on a travaillé ensemble pendant environ une demi-heure.
    \end{itemize}
    \subsection{Checklist}
    \begin{center}
        \begin{tabular}{|c|c|c|c|}
            \hline
            Numéro de la tâche & Effectuée & Cătălin & Liviu \\
            \hline
            \hline 1 & x &   & x \\ 
            \hline 2 & x & x &   \\
            \hline 3 & x & x &   \\
            \hline 4 & x & x & x \\
            \hline 5 & x & x & x \\
            \hline
        \end{tabular}
    \end{center}
    \newpage

    \section{Rendu 2 (Laboratoire 2 : 21.03.2024) }
    \subsection{Tâches à effectuer}
    Modifier le programme qui se trouve dans la ROM en utilisant uniquement le jeu d'instructions fourni. Un testbench doit être créé afin de valider et vérifier le bon fonctionnement de programme. \\
    L'application en assembleur doit torner en boucle et a les fonctionnalités suivantes:
    \begin{enumerate}
        \item Lire l'état des 8 dilswitch 1 et les sauvegarder en RAM
        \item Lire l'état des 8 dilswitch 2
        \item Si l'état des 8 dilswitchs 2 sont 0 alors:
        \begin{enumerate}
            \item Afficher la valeur binaire lue sur les dilswitch 1 sur les 8 leds du bargraphe 1
            \item Afficher sur l'affichage 7 segment la valeur hexadécimale correspondant à l'état des 4 bits de poids forts des dilswitch 1 
        \end{enumerate}
        \item Sinon:
        \begin{enumerate}
            \item Faire un ou exclusif entre les 8 dilswitch 1 et les 8 dilswitch 2, afficher le résultat sur le bargraphe 1 et afficher un C(pour calcul) sur l'affichage 7 segments
        \end{enumerate}
    \end{enumerate}
    \subsection{Parcours}
    \begin{itemize}
        \item \textbf{21.03.2024:} \\
        \underline{Ensemble:}
        \begin{enumerate}
            \item On a regardé l'ensemble du project, afin de mieux comprendre le fonctionnement du processeur.
            \item On a revu la table de vérité (Rendu 1) du séquenceur, parce qu'elle n'a pas été bien faite.
        \end{enumerate}
    \end{itemize}
    \subsection{Checklist}
    \begin{center}
        \begin{tabular}{|c|c|c|c|}
            \hline
            Numéro de la tâche & Effectuée & Cătălin & Liviu \\
            \hline
            \hline 1 &   &   &   \\ 
            \hline 2 &   &   &   \\
            \hline 3.a &   &   &   \\
            \hline 3.b &   &   &   \\
            \hline 4.a &   &   &   \\
            \hline
        \end{tabular}
    \end{center}
    \newpage

    \section{Rendu num (Laboratoire num : date) }
    \subsection{Tâches à effectuer}
    \begin{itemize}
        \item Task1
    \end{itemize}
    \newpage
\end{document}
